\documentclass[a4paper,12pt]{article}

% very small margins (around 1.5cm)
% disable package geometry if you use this
\usepackage{fullpage}

% some mathematical symbols
\usepackage{amsmath}

% mathematical symbols
% conflicts with package program
\usepackage{amssymb}

% url addresses using \url{http://domain.com/}
\usepackage{url}

% use \includegraphics[scale=1.00]{file.jpg} for images
\usepackage{graphicx}

% for proper placement of floating figures
\usepackage{float}

\usepackage{multirow}
%\usepackage{pseudocode}
%\usepackage{booktabs}
%\usepackage{color}
%\usepackage{algorithmic} %pseudocode
%\usepackage{algorithm} %pseudocode as whole procedures
%\usepackage{program} %more sophisticated pseudocode

\begin{document}

\title{Static fluid~flow~simulation using~the~finite~element~method}
\date{January 10, 2014}
\author{Mateusz Bysiek}
\maketitle

\begin{footnotesize}

Written by Mateusz Bysiek of Computer~Science, Faculty~of~Mathematics~and~Information~Science,
Warsaw~University~of~Technology.

bysiekm at student dot mini dot pw dot edu dot pl

\end{footnotesize}

\section{Task}

Write a computer program that implements finite element method for static fluid flow, for uniorm triangular and
rectangular meshes. Use Laplace's equation.

\section{Theoretical introduction}

\subsection{Static fluid flow}

Static fluid flow can be understood as the potential of how the water will flow given some initial conditions. It is
important to note that the task of static fluid flow simulation is equivalent to static heat transfer simulation. And as
such, it can be understood as how the temperature will be distributed across space given some initial conditions.
\cite{wiki_heat_transfer}

For example:


We have a generally ``neutral'' environment, for example some place few meters underground. Plus, we have some
relatively hot places in our environment. They can be any source of heat, for example some pipes with hot water. Also,
we have some relatively cold places in our environment. They can be any hypothetical devices that suck in heat, or just
some pipes with coolant, or cold water. What characterizes such environment is that it is static -- the water in the
pipes always has the same temperature.

Using static flow calculations, we can predict exaclty what temperature will be at each point in our environment. You
may notice, that normally such predictions would require you to make calculations for each point, and since heat
propagates in the ground, after you calculate temperature at some point you have to adjust temperature at all
neighbouring points. This leads to either infinite calculations, or if you know how to handle such problems it leads to
partial differential equations.

Differential equations are solvable, but first of all are complicated, so they require large computational power. Second
of all, you rarely need to know exact temperature in every place around. Usually some regions are of more interest than
others, and more importantly, even in these regions you probably don't need to know the temperature at every single
point.

\subsection{Finite element method}

And that is where the finite element method comes in. For example, for you may want to know the approximate temperature
of every $m^2$ of the ground. You can divide your whole environment into fragments called \emph{elements}. In our
example, you may call each square meter the \emph{element}. \cite{wiki_finite_element_method}

To determine what temperature is at each location, is the same as to solve the Laplace's equation for given conditions.
\cite{wiki_laplaces_equation}

\[ \Delta \phi =  \nabla^2 \phi = 0 \]

Specifically, in 2 dimensions the equation has a certain form.

\[ \frac{\partial^2 \psi}{\partial x^2} + \frac{\partial^2 \psi}{\partial y^2} \equiv \psi_{xx} + \psi_{yy} = 0 \]

This means that our problem is a very specific case, and even partial differential equations are simplified.  

\subsubsection{Triangular elements}

Trianglar elements are those that have 3 vertices.

\subsubsection{Rectangular elements}

Rectangular elements are those that have 4 vertices.

\section{Implementation}

Computational part is implemented in C++, user interface is made using Qt, while the visualization itself is made using
OpenGL.

\subsection{Technologies}

The FEM computations are implemented in C++11. Full list of technologies used for computation:

\begin{itemize}
  \item C++11 -- backbone of the implementation
  \item Boost uBLAS (Basic Linear Algebra Library) -- part of Boost dedicated to linear algebra and matrix operations
\end{itemize}

The visual part of the solution is implemented using OpenGL. Full list of technologies used for visualization:

\begin{itemize}
  \item Qt -- for GUI (Graphical User Interface)
  \item OpenGL -- backbone of 3D visualization
  \item freetype -- for FreeType fonts displayed in OpenGL environment
  \item ftgl -- for displaying text in OpenGL environment.
\end{itemize}

Other general-purpose technologies used:

\begin{itemize}
  \item Doxygen -- for technical documentation
  \item LaTeX -- for theoretical documentation and user guide
  \item libxml2 -- for file I/O
\end{itemize}

\subsection{Algorithm}

The program starts with measuring the most important parameters of the mesh:

\begin{itemize}

  \item Number of vertices in mesh is $N$.

  \item Total number of elements (i.e. triangles or rectangles) is $E$.

  \item Number of vertices in each element of the mesh is $V$ -- for triangular elements $V$ is $3$, for rectangular it
  is $4$.

  \item Number of vertices that have defined boundary condition is $B$.

\end{itemize}

Three main matrices are creates:

\begin{itemize}

  \item A stiffness matrix $S$, of size $N \times N$.

  \item A mass matrix $M$, of size $1 \times N$ -- in other words a $N$-dimensional vector.

  \item A result matrix $R$, of size $1 \times N$

\end{itemize}

All of the values in matrices $S$ and $M$ are set to zero.

Then, a supporting matrix, local stiffness matrix $L$ of size $V \times V$ is created.

$L$ is set to precalculated results of two-dimensional integration so that for each element the correct ``shape'' of
importance of each vertex is applied.

\begin{itemize}

  \item For triangulr elements:
  \[
  \left| \begin{array}{rrr}
  %\left[ \begin{aligned}
  -0.18 &  0.09 &  0.09 \\
   0.09 & -0.17 &  0.08 \\
   0.09 &  0.08 & -0.17 \\
  %\end{aligned} \right]
  \end{array} \right|
  \]

  \item For rectangular elements:
  \[
  \left| \begin{array}{rrrr}
   0.666667 & -0.166667 & -0.333333 & -0.166667 \\
  -0.166667 &  0.666667 & -0.166667 & -0.333333 \\
  -0.333333 & -0.166667 &  0.666667 & -0.166667 \\
  -0.166667 & -0.333333 & -0.166667 &  0.666667 \\
  \end{array} \right|
  \]

\end{itemize}

After that, iteration over all $E$ elements starts. For each element the same below procedure is repeated:

\begin{itemize}

  \item a ``multiplier'' is calculated from the determinant of the element. That ``multipier'' depends directly on the
  area of the element, i.e. if all elements of the mesh have the same size, all ``multipliers'' are identical.

  \item to the stiffnes matrix $S$, the partial results of $L$ times the ``multiplier'' are added.

\end{itemize}

Then, iteration over all $B$ vertices that have boundary condition starts. For each such point the following is done:

\begin{itemize}

  \item Each vertex has a corresponding row of the stiffness matrix $S$, since height of $S$ is equal to $N$. The row
  that corresponds to current $i$th boundary-conditioned vertex is set to the vector$(0,\ldots,1,\ldots,0)$ where the
  $1$ is placed at $i$th place. Incidentally it is on the diagonal of the stiffness matrix.
  
  \item Each such vertex also has a corresponding element of of mass matrix $M$, since its height is also equal to $N$.
  The value at $i$th place of $M$ is set to the boundary condition (i.e. the ``z'' coordinate of the vetrex).

\end{itemize}

At this point, the stiffness matrix $S$ and mass matrix $M$ are both ready, and the resulting matrix $R$ is calculated
by left division of $S$ by $M$.

\section{Examples}

\subsection{Trivial}

In the trivial case, where there are no conditions defined at all, the software is behaving nicely by not altering the
input with any noise.

\subsection{Bounary conditions}

If the whole border of the mesh has defined boundary conditions, the software of course manages to compute the final
result. For most reasonable mesh sizes (up to approx. 15 x 15) the computation happens in real-time.

\subsection{Free boundaries}

The program manages to compute correctly in situations where some part of border has no boundary conditions.

\subsection{Several other examples}

I would like to include some nice examples of smoothened planes, which in the end inspired the name of the application. 

\begin{thebibliography}{9}

\bibitem{wiki_heat_transfer}
  Heat transfer,
  Wikipedia,
  \url{https://en.wikipedia.org/wiki/Heat_transfer},
  accessed 2014-01-22

\bibitem{wiki_fluid_statics}
  Fluid statics,
  Wikipedia,
  \url{https://en.wikipedia.org/wiki/Fluid_statics},
  accessed 2014-01-22

\bibitem{wiki_finite_element_method}
  Finite element method,
  Wikipedia,
  \url{https://en.wikipedia.org/wiki/Finite_element_method},
  accessed 2014-01-22

\bibitem{wiki_laplaces_equation}
  Laplace's equation,
  Wikipedia,
  \url{https://en.wikipedia.org/wiki/Laplace's_equation},
  accessed 2014-01-22

%https://en.wikipedia.org/wiki/Stiff_equation
%https://en.wikipedia.org/wiki/Stiffness_matrix
%https://en.wikipedia.org/wiki/Mass_matrix
%https://en.wikipedia.org/wiki/Division_%28mathematics%29#Of_matrices

\end{thebibliography}

\end{document}
